\documentclass[12pt, letterpaper]{article}
\usepackage[T1]{fontenc}
\usepackage[margin=1in]{geometry}
\usepackage{amsthm}


\title{Physics 1D03 Lecture 2: Kinematics in One Dimension}
\author{Talha Ahmad}
\date{September 6, 2023}

\begin{document}

\maketitle

\section{Dimensional Analysis}

There are 3 fundamental dimensions in mechanics:

\begin{itemize}
	\item Mass (Denoted by "M")
	\item Length (Denoted by "L")
	\item Time (Denoted by "T")
\end{itemize}

The dimensions across a single equation must be \underline{consistent}.
Some basic rules regarding dimensions in an equation:

\begin{enumerate}
	\item Terms on both sides of an equation must have same dimensions.
	\item Log and exponents apply to \textbf{dimensionless} quantities.
	\item Trig functions apply to \textbf{dimensionless} quantities (ie. radians or degrees).
	\item Constants cannot be determined through dimensional analysis.
\end{enumerate}

Dimensional analysis analyzes the relationship between physical quantities by taking their
fundamental dimensions (M, L, T) in addition to their units of measurement and tracking these
quantities.

\bigskip

This is an example for how the units of \emph{E} in Einstein's famous energy-mass equialence,
$E = mc^2$, can be derived through dimensional analysis:

\begin{center}

	$E = mc^2$ 
	
	In this formula, \textit{m} is known to be mass; and $c^2$, which is the speed of light,
	is known to be velocity squared. Velocity is known to be $\frac{L}{T}$.

	\medskip

	\LARGE $\frac{M \cdot L^2}{T^2}$ \normalsize

	\bigskip

	Over here, we have mass multiplied by length squared all over time squared.
	These dimensions are equal to energy; therefore, the units of \textit{E}
	are joules.

\end{center}

\bigskip

\noindent
Another example in which we determine if an equation is correct based solely on its dimensions:

\begin{center}
	$x = x_0 + v_0t^2 + at$

	\medskip

	When broken down into its fundamental diimensions, this equation is of type:

	\medskip

	$L = L + LT + \LARGE \frac{L}{T} \normalsize$

	\medskip

	Therefore, this equation is not valid since all terms \underline{need} to be of fundamental dimension
	type "length," yet one is of type length times time, while the other is of type length over time.
	To fix this issue, we could divide the second term by \textit{T} and multiply the third term by \textit{T},
	which would result in all the terms being \textit{L}. This would be akin to the overall expression being:

	\medskip

	$x = x_0 + v_0t + at^2$
	
	\medskip

	While this equation is valid from a dimensional standpoint, dimensional analysis cannot determine any constants,
	and as we know this kinematic equation to be, it is originally of form:

	\medskip

	$x = x_0 + v_0t + \frac{1}{2}at^2$

\end{center}

\bigskip

\noindent
Through dimensional analysis, we are also capable of deriving laws of physics. As an example, consider a pendulum
with string length \textit{L} and of mass \textit{M}. In order to derive an equation for the period of a pendulum 
we know that periods are of fundamental dimension type \textit{T}. We know that \textit{g} acceleration is acting
on the pendulum, causing it to swing. Since acceleration is of form $\frac{L}{T^2}$, we know that dividing \textit{L}
by $\frac{L}{T^2}$ will result in $T^2$. Since we need a final answer of type \textit{T}, we can take the square root
of this. Therefore, the final equation derived to find the swing of a pendulum is:

\begin{center}
	$P = C\sqrt{\frac{L}{g}}$
\end{center}

\noindent
Where C is a constant that cannot be determined through dimensional analysis.

\section{Kinematics}

Kinematics is the study of motion not including what causes or modifies that motion. Put simply, it is the study
of motion itself without external factors acting upon it. The object or particle of motion that is under study
is referred to as a \textit{\textbf{"model"}} and when this particle's motion occurs along a straight line, the motion is
\textit{\textbf{"one dimensional"}}.

One-dimensional motion is described by scalars (numbers with units and no direction) as functions of time. These
functions are:

\begin{itemize}
	\item position: $x(t)$
	\item velocity: $v(t)$
	\item acceleration: $a(t)$
\end{itemize}

\noindent
In regards to a position function specifically, there are two values of measurement of utmost importance:
displacement and distance. Displacement refers to the difference between the starting and ending position
values at two separate points across the function whereas distance refers to the total distance of the
particle's path.

Velocity is the rate at which position is changing over time, and acceleration is the rate at which velocity
is changing over time. If we calculate the slope between two separate points in a position-time graph,
we end up with the average velocity a particle has over that point. If we instead move both points closer,
and closer together, $\lim_{\Delta x\to0}$, we end up with a tangent line to the graph that tells us the instantaneous
velocity of the particle at that specific moment in time. By taking the $\frac{d}{dx}$, or derivative, of each item, 
we can obtain the next one.

\medskip

The opposite is also true. In order to go from a velocity-time graph to a displacement
graph, we can just add up the individual velocities over a certain period of time.
In other words:

$$ \lim_{t\to0} \sum_iv(t_i)\Delta t $$

Which means that we are treating the velocity-time graph as a series of tiny, tiny
rectangles, as denoted by the width of the rectangle ($t$) which is close to 0.
This gives us the area underneath the velocity-time graph and in doing so, allows
us to create a displacement-time graph. This equation can be simplified to:

\large $$\int_{t_1}^{t_2}v(t)dx$$ \normalsize
\end{document}
